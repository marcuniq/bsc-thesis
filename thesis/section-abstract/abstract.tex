\begin{abstract}
Recommendation systems have become omni-present: we read online news articles, buy books from e-commerce websites and watch videos from online streaming services, which are all very likely to be suggested to us by a recommendation system.
The most common applied algorithm utilizes the collaborative filtering technique, which makes only use of the user-item rating matrix. This thesis introduces two approaches, which make use of extra data about the items.
One of our proposed models, MPCFs-SI, is based on a nonlinear matrix factorization model for collaborative filtering (MPCFs), and uses the extra data to regularize the model.
The second model we propose, MFNN, is an ensemble of a matrix factorization and a neural network and makes direct use of the extra data.
We show that MPCFs-SI slightly outperforms baseline recommender on a subset of both MovieLens 100k and 1M datasets.
Our second model MFNN is always performing worse that the MPCFs model.
\end{abstract}
