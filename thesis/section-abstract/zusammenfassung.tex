\begin{zusammenfassung}
Der meist angewandte Algorithmus für Empfehlungssysteme basiert auf kollaborativem Filtern, welcher dazu nur die Bewertungen von Objekten durch Benutzer verwendet.
Diese Arbeit präsentiert zwei neue Ansätze wie zusätzliche Daten in Form von Merkmalsvektoren eingesetzt werden können, um bessere Top-N Empfehlungen abzugeben.
Unser erstes Modell, MPCFs-SI, basiert auf einer nichtlinearen Matrix Faktorisierung (MPCFs) und verwendet die zusätzlichen Daten um das Modell zu regularisieren.
Das zweite Modell, MFNN, ist eine Kombination aus Matrix Faktorisierung und einem neuralen Netz und spielt die zusätzlichen Daten als Eingabe ins neurale Netz ein.
Unsere Experimente haben gezeigt, dass MPCFs-SI bessere Leistungen erbringt als das beste Vergleichsmodell MPCFs auf unseren MovieLens 100k und MovieLens 1M Teildatensätzen.
MFNN ist schlechter als das Vergleichsmodell MPCFs auf dem MovieLens 100k Teildatensatz, jedoch ist es auf einem ähnlichen Performanzlevel wie MPCFs-SI auf dem grösseren MovieLens 1M Teildatensatz.
\end{zusammenfassung}
