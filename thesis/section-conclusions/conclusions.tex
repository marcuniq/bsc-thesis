\chapter{Conclusions}
\label{c:conclusions}

In this work, two models, MPCFs-SI and MFNN, were developed which incorporate side information encoded as feature vectors into recommender systems in order to improve top-N recommendations.
MPCFs-SI utilizes the side information to regularize its underlying base recommender, a nonlinear matrix factorization model (MPCFs) developed by \cite{Kabbur2015}.
MFNN is a combination of a matrix factorization and a neural network.
That model uses the additional extra data as an input to the neural network, together with the user and item latent vectors.

Our results have shown that MPCFs-SI is able to outperform the baseline recommender MPCFs on the two MovieLens subsets, ML-100k-sub and ML-1M-sub, for the top-N recommendation task.
MFNN is inferior to MPCFs on the ML-100k-sub dataset, however, it was at a similar level of performance as MPCFs-SI on the ML-1M-sub dataset.
In particular, MPCFs-SI has an advantage for users with less than 10 training ratings on ML-1M-sub, while MFNN is superior to other recommender for users with less than 200 training ratings also on ML-1M-sub.

Future research efforts should focus on improving the Doc2Vec prediction of the MPCFs-SI model by using a separate weight matrix $\mathbf{G}$ per item.
For the MFNN model a similar strategy of having separate weights $\mathbf{w}, \mathbf{W^\prime}$ per user or item could be applied and might lead to further improvements.