\chapter{Introduction}
\label{c:introduction}

Recommender systems are software systems suggesting items to users which might fit their personal tastes.
Such systems play a crucial part in e-commerce, social networks, news, music and video websites.

This work investigates methods which make use of data about the users, the items and ratings.

In Chapter \ref{c:incorporating-extra-data} we present two ways of incorporating extra data into recommender systems.
The evaluation of the proposed methods are compared with state-of-the-art methods in Chapter \ref{c:evaluation}.
The first section of this chapter gives an a brief overview over different methods for recommending systems.
We further discuss related work in Section \ref{st:related-work} and conclude this chapter with our motivation for this work.

\section{Background}
\label{st:background}

Methods which try to predict valuable items for a given user can be broadly classified into two groups: collaborative filtering (CF) based methods and content based methods.
These methods differ in what kind of input data they need.
Collaborative filtering methods make use of the user-item rating matrix in order to build models \cite{Kabbur2014}.
Among all the CF algorithms, the most successful ones are the latent factor models \cite{Bao2014}.
These models try to explain user ratings by characterizing both items and users on factors inferred from the rating patterns \cite{Bao2014}.
Content based methods on the other hand make use of meta data about the items as well as the users.
A combination of several methods is referred to as hybrid recommender systems \cite{Ricci2011}.
The purpose of such hybrid systems is to use the advantages of one method and fix the disadvantages of another \cite{Ricci2011}.

While CF techniques are very popular and widely used, one of its main disadvantages is known as the cold start problem \cite{Christoffel2014}.
Collaborative filtering methods require new users to rate some items before the system can suggest items, and new items need to be rated before it can be recommended to users.
In case of the cold start problem, the rating history on a user or item is not sufficient to generalize well to future preferences.
Often though, other sources of information provide richer cues on cold start users or items.


\section{Related Work}
\label{st:related-work}
In this section, we review related approaches to our work.

Simultaneously considering the ratings and the accompanied review text was proposed in \cite{Bao2014}.
The presented model, called \textbf{TopicMF}, learns a recommender model by combining latent factor modeling of rating data with topic modeling in user-review texts.
\cite{Bao2014} show on 22 real-world datasets that their method handles the data sparsity better than state-of-the-art latent factor models.

\cite{Almahairi2015} also have shown that utilizing additional data as a way of regularizing the derived item representations improve the generalization performance of a rating prediction model.
While \cite{Bao2014} have used topic modeling for the user-review text, \cite{Almahairi2015} proposed a distributed bag-of-word approach \textbf{BoWLF} and a recurrent neural network approach \textbf{LMLF} to model the user-review texts.


\section{Motivation}
\label{st:motivation}

As we have discussed in the previous section, recent work has shown that better rating predictions can be obtained by incorporating text-based side information.
Motivated by these recent successes, here we explore alternative approaches to exploiting side information for top-N recommendations.
Specifically, we study how feature vectors extracted from movie subtitles can be used to improve the performance of a latent factor model.

We introduce two approaches to incorporate side information into recommender and compare these to state-of-the-art-based approaches \cite{Kabbur2015,Ning2011,Rendle2009}.
 